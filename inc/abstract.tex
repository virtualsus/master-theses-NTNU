\chapter*{Abstract}

%Present the project topic and the need for the work.
%State the specific objectives of the project.
%Re-cap the approach taken, major decisions and results.
%Summarize the major conclusions and recommendations of your work.



Free-hand gesture interface is a new and exciting field of Human Computer Interaction. Due to the fact that gestures are natural, there is no better tool for manipulating all kind of computer systems. Such interfaces are already employed in industry and health care and are especially beneficial when visual attention is required as well as in aseptic environments. Disadvantages may occur when for instance, there are no guidelines and actions are hard to predict or when a user suffers from motor performance deficiencies. On the other hand, using gestures to manipulate interfaces can be a fun and enjoyable experience. 

The topic of this master thesis is about exploring the possibility of using free-hand gesture interaction as a screening method to measure fine motor performance skills in children. The ulterior motive of this concept was to create a system to determine motor performance deficiency and identify development delays such as those that can be found in Autism.
In order to realize the project, a motion sensor was purchased that could detect hand gestures. The Leap Motion controller was a perfect choice because of its economical benefits and it's ability to record midair hand and finger gestures. The point of departure for creating this concept was a motor coordination test called Movement Assessment Battery. This is a standard performance test that includes specific motor exercises. Ideas from this test were selected to create a concept of 5 mini games. However only one game was selected and implemented further due to the limited capacity. The game "Mole in the hole" was about using gestures to manipulate and navigate the mole through its tunnel. The point of this game was to explore if it is possible to extract parameters that could be used to measure motor performance. The game was tested on 8 children and data about time and accuracy was collected. Time and accuracy were found to be useful parameters to measure motor performance. The ability test demonstrated that children under 5 years of age had the most trouble carrying out the test because of their limited fine motor ability. 
The study showed that this kind of interaction was of great enjoyment for the participant. However, when children are the target group, an appealing graphical interface is essential as well.  
To sum up, the findings indicate that it is possible to use a free-hand gesture based concept to measure motor performance in children. However, it should rather be used as a supplementary tool in a complex system for determining development delays.\\

The information obtained in this study contributes to a basic introduction and understanding of a complex research area on screening methods for development delays. It is recommended to continue the research with a larger sample in order to gather more statistical data to examine motor performance.  





\hypersetup{pageanchor=false}