\chapter*{Abstract}

%Present the project topic and the need for the work.
%State the specific objectives of the project.
%Re-cap the approach taken, major decisions and results.
%Summarize the major conclusions and recommendations of your work.



Free-hand gesture interface is a new and exciting field of Human Computer Interaction. Due to the fact that gestures are natural, there is no better tool for manipulating systems. Such interfaces are already employed in industry and health care and are especially beneficial when visual attention is required and also in aseptic environments. Disadvantage may occur when for instance, there are no guidelines and actions are hard to predict or when a user suffer from motor performance deficits. On the other side, using gesture to manipulate interfaces can be a fun and enjoyable experience. 

The topic of this master thesis is about exploring the possibility of using free-hand gesture interaction as a screening method to measure fine motor performance in children. The ulterior motive of this concept was to create a system to determine motor performance deficit and identify development delays such as it can be found in Autism.
In order to realize the project, a motion sensor was purchased that could detect hand gestures. The Leap Motion controller was a perfect choice because of its economical benefits and due to the fact that it is able to record mid air hand and finger gestures. The point of departure for creating a concept was a motor coordination test called Movement Assessment Battery. This is a standard performance test that includes specific motor exercises. Ideas from this test where selected to create a concept of 5 mini games. However only one game was select and implemented further due to the limited capacity. The game "Mole in the hole" was about using gestures to manipulate and navigate the mole through its tunnel. The point with this game was to explore if it is possible to extract parameters that could be use to measure motor performance. The games was tested on 8 children and it was possible to gather data about time and accuracy. Time and accuracy were useful parameters in order to measure motor performance. The ability test demonstrated that children under 5 years of age had most trouble to carry out the test because of their limited fine motor ability. 
The study showed that this kind of interaction is of great enjoyment. However, when children are target group an appealing graphical interface is essential as well.  
To sum up, the findings indicate that is is possible use a free-hand gesture based concept to measure motor performance in children. However, it should rather be used as a supplementary tool in a complex system for determining development delays.\\

The information obtained in this study contribute to a basic introduction and understanding to a complex research area on screening methods for development delays. It is recommended to continue the research with a larger sample in order to gather more statistical data to examine motor  





\hypersetup{pageanchor=false}