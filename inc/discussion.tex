\chapter{Discussion}
\label{chap:discussion}
%theory
%data
%proving the argument

%best age group
%the deviation of time and hit was as expected varying. the study was not about reliable parameter but if it is possible to gather parameters. since the game just was a prototype the result was expected not to be reliable

%She also supported the pointing hand with her other hand, not because it was hard to hold the hand up but rather to help the finger in the right direction. Something like that could be used to point to a hand-eye coordination issue.

%critical exam findings
%contribution
%compare with other work, facts
%make a knowledge claim
%observation backup with evidence!!!!
%any unexpected findings
% weaknes of the study , eks choice and amount of particiant , bad code, inexperience interviewer, 

%validity + reliability

%can not be used to determine early stage of ASD > refer age of child in test

%Earlier, we pointed out that....However...
%paramenters + observation > ASD
%ikke strekke > konsentrasjons ?

The literature review, pointed out that designing gestural touch-less interfaces demand guidelines and an understanding of the challenges it arises (Norman Nielsen 2010, Malixia, Montero 2010). Researcher are two minded whether gestural interaction is natural and in what extent it is a convenient form for interacting with computer and other systems(Montero 2010, Maliza2012, Norman 2010, Park 2013, riener 2012). Other argue, that gesture interfaces are stimulating and enjoyable (Loehnman 2013, Ren 2013) Moreover, navigating precisely in free space is argued as a big disadvantage when dealing with gesture interfaces (Ni et al, 2011, Siek, Roger, Conelley).

The analyzed data presented in the previous chapter revealed some interesting findings that shows a clear correlation between the literature review and the prove of principle type of review. Although the gathered data corresponds to a small sample of the population it provided valuable knowledge about hand-free gesture based screening systems. As Norman, Nielsen and co stated, gesture based interaction need guidelines and do not come natural if the user never needed to deal with this kind of interaction. The children in this study would not know how to play the game if it had not been showed them. They never had seen this kind of interaction before and for that reason could not build a conceptual model of the system (Norman, 2002). However, if gesture interaction would be more usual such as touch based gesture interaction than it would be more know and more natural to people. Nevertheless, without guidelines that define what type of gesture trigger a certain action, it would be a challenging form of interaction for all unknown systems. 
The study also confirmed that gesture interfaces offers a pleasurable experience (Loehnman, REn). The children valued the games highly which also was observed by their body language during the task. This could be explained by the fact that none of them had experience with free-hand gesture based games and it confirmed that a appealing graphic has a motivating factor (charsky, 2010)(Paraskevopoulo).
The problem with navigating in free space as Ni and Siek argued, can be confirmed only partly. The study showed that users who have not a fully developed fine motor ability such as in toddlers and very young children have experienced such challenge. However, the fine motor ability increases with age (Figure \ref{fig:relationTimeAge}). Nevertheless, it can not be ruled out that people with fine motor deficit or eye - hand coordination deficit will also experience problems with navigating precisely since such task demand high fine motor performance. This assertion can be confirmed by the observation of a participant that needed to support her pointing hand in order to be more precisely in her navigation and pointing action. 
The above mentioned issues refer to the second research question where the interest focused on how toddlers and children cope with hand-free gesture interaction. The result indicates that hypothesis 2 applies for children older than 5 years of age.

As implied in chapter \ref{chap:aim}, figure \ref{fig:tableOfModules}, in order to determine motor performance skills
the desired primary parameters from the gamed "Mole in the hole" were time (race time) and number of hits (path deviation). The secondary parameters focused more on the ability of doing a gesture (touchzone, pointable.extended). The parameters could be successfully extracted as illustrated in chapter \ref{chap:dataanalysis}. The time taking starts by extracting the finger and stops when the mole reaches the finish line. Hits were counted for each time the mole deviated from the defined path. Good motor performance skills are indicated by time and hits values. An increase of those values, means a deterioration of motor performance which could indicate a motor performance deficit. However, in order to be able to determine motor performance deficit, a standard value is need. Data could then be compared with a standard and it may be drawn a conclusion whether there is deficit or not. Unfortunately, identifying a standard value for normal motor performance is beyond the scope of this project. 
By proving that it is possible to extract parameters form the Leap Motion device that could be used to determine motor performance deficit, justifies hypothesis 1. 

The concept of this study was shown and discussed with a pediatrician located in Trondheim and specialized in child health. It was agreed that her identity will not be revealed in this project. However, a professional opinion was of great value for this project, particular in relation to the third research question which is about reliability and whether it is a appropriate test concept. Unfortunately, the pediatrician was very cautious in her statements since she believed that this area belongs rather to a specialist such as neurologist. Nevertheless, she had some interesting statements - such as, for instance: \textit{"The idea is exciting."} or \textit{"It seems a little difficult."} She mentioned also that this concept could theoretically be used to discover many different deficits. She explained that fine motor problems are not necessarily the main essence in Autism disorder and sometimes it just takes some time a child to develop, for that reason, this concept should only be a part of a more complex screening method. She also said that neurological examinations follow a checklist to test different nerves, fine and cross motor performance and various reflexes. Those test are usual evaluated observations and therefor subjective. For that reason, she said, is this concept a good idea since this i more objective. \textit{"We could benefit from such a fine motor diagnostic tool".} 
The study showed that it is possible to extract objective parameters, the observation and follow up question confirmed great enjoyment and the interview with the pediatrician verified the need of reliable and objective screening methods. To conclude, with the before-mentioned facts hypothesis 3 can be thus confirmed. 




\section{Weakness and limitation}


To point out the weaknesses of this work, is part of the understanding what limitations occurred during the project.
Earlier it was mentioned that Autism Spectrum Disorder is used as an example for motor performance deficit. This turned out not be the best example since Autism is a complex disorder. The interviews showed that people were a somewhat skeptical about using this concept for autism screening even if the literature review revealed something else. Having a small sample size and a restricted demographic sample were also a weak point in this study. Although the sample size provided enough observational data it was not statistical presentable though. This study relied much on observation and interview, additionally, being an inexperienced moderator and at the same time a note-taker could have led to subjective interpretation of the data. Furthermore, due to the limited amount of time combined with little coding practice, the prototype was more or less inadequate. For instance, there was no restriction of where the mole was allowed to go and where not, thus a direct start - finish run was possible. Another weakness that was observed as well, was that sometimes the mole did not move smoothly as it should have which resulted in minor frustration of some participants.
Finally, it might have been beneficial to discuss the concept with a physician specialized in neurology in order to investigate and discuss more extensively disorders connected to motor performance and to get a more professional opinion on the concept suggested in this study. 
