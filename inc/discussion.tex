\chapter{Discussion}
\label{chap:discussion}
%theory
%data
%proving the argument

%best age group
%the deviation of time and hit was as expected varying. the study was not about reliable parameter but if it is possible to gather parameters. since the game just was a prototype the result was expected not to be reliable

%She also supported the pointing hand with her other hand, not because it was hard to hold the hand up but rather to help the finger in the right direction. Something like that could be used to point to a hand-eye coordination issue.

%critical exam findings
%contribution
%compare with other work, facts
%make a knowledge claim
%observation backup with evidence
%any unexpected findings
% weaknes of the study , eks choice and amount of particiant , bad code, inexperience interviewer, 

%validity + reliability

%can not be used to determine early stage of ASD > refer age of child in test

%Earlier, we pointed out that....However...
%paramenters + observation > ASD

The literature review, pointed out that designing gestural touch-less interfaces demand guidelines and an understanding of the challenges it arises (Norman Nielsen 2010, Malixia, Montero 2010). Researcher are two minded whether gestural interaction is natural and in what extent it is a convenient form for interacting with computer and other systems(Montero 2010, Maliza2012, Norman 2010, Park 2013, riener 2012). Other argue, that gesture interfaces are stimulating and enjoyable (Loehnman 2013, Ren 2013) Moreover, navigating precisely in free space is argued as a big disadvantage when dealing with gesture interfaces (Ni et al, 2011, Siek, Roger, Conelley).

The analyzed data presented in the previous chapter revealed some interesting findings that shows a clear correlation between the literature review and the prove of principle type of review. Although the gathered data corresponds to a small sample of the population it provided valuable knowledge about hand-free gesture based screening systems. As Norman, Nielsen and co stated, gesture based interaction need guidelines and do not come natural if the user never needed to deal with this kind of interaction. The children in this study would not know how to play the game if it had not been showed them. They never had seen this kind of interaction before and for that reason could not build a conceptual model of the system (Norman, 2002). However, if gesture interaction would be more usual such as touch based gesture interaction than it would be more know and more natural to people. Nevertheless, without guidelines that define what type of gesture trigger a certain action, it would be a challenging form of interaction for all unknown systems. 
The study also confirmed that gesture interfaces offers a pleasurable experience (Loehnman, REn). The children valued the games highly which also was observed by their body language during the task. 
The problem with navigating in free space as Ni and Siek argued, can be confirmed only partly. The study showed that users who have not a fully developed fine motor ability such as in toddlers and very young children have experienced such challenge. However, the fine motor ability increases with age (Figure \ref{fig:relationTimeAge}). Nevertheless, it can not be ruled out that people with fine motor deficit or eye - hand coordination deficit will also experience problems with navigating precisely since such task demand high fine motor performance. This assertion can be confirmed by the observation of a participant that needed to support her pointing hand in order to be more precisely in her navigation and pointing action. 

As suggested in chapter \ref{chap:aim}, figure \ref{fig:tableOfModules}, the main parameters to be extracted from the game Mole in the hole were time - needed to finish the game and number of hits - path deviation. The secondary parameters where touchzone and pointable extended. 
%ikke strekke > konsentrasjons ?






\section{Limitation}

%what is left out, yet to-do
