\chapter{Background, theory and existing literature}
\label{chap:background}


\section{Keywords}
\label{sec:keywords}
Index terms, so-called keywords, are of interest to research an information retrieval. A list of keywords is made to narrow down the search area for this project. The list includes the following 8 keywords: 
\newline

\textit{Freehand Gesture Interaction, Human Computer Interaction, Fine Motor Skills, Leap Motion, Development Delays, Autism Spectrum Disorder, Development Screening Methods, Gesture Recognition}
\newline

Each keyword and combination of keywords was the basis of my search within the digital libraries of Oria, ACM, SpringerLink, ScienceDirect and IEEE.

  

\section{Gesture definition, guidelines and interaction}
\label{sec:gesture}

Gestures are very common in intercommunications between humans. Arm, face and head movements are gestures that are evoked by an idea, emotion or reaction \cite{Grundmann2016}. For a better understanding, gestures accompany verbal information. However in some situations are gestures more suitable than speech, such as in noisy environments \cite{Wagner2014}.
Some HCI researcher classify and categorize gesture in order to differentiate them from each other. McNeill for instance, categorized gestures into gesticulation, pantomime, emblem, sign language. He then classified gesticulation into iconic, metaphoric, rhythmic, cohesive, and deictic \cite{McNeill205}. Pavlovic et al. subdivides gesture into communicative (gestures that have to be learned) and natural hand and arm movements, called manipulative gestures \cite{Pavlovic1997}. Park and Han distinguished between manipulative and communicative gestures, which are a part of a nonverbal component of speech \cite{ParkWonkyu2013}. Furthermore, an interesting claim made Riener. He pointed out that gestures might differ from gender, ethnicity and cultural background \cite{Riener2012}. 
\newline 
Gestures can be part of a multimodal application, which can be a solution for people with challenges in everyday life in order to compensate impairments such as speech or hearing disability. Anastasiou carried out a study where a subject navigated a wheelchair through a set-up of smart home environment by only using speech and gestures. Her study showed that the subjects mostly used gesture when something went wrong \cite{Anastasiou2012}. Another study, relevant for this project was conduct by van Beurden et al., which compared gestured-base investigation to device-based interaction  and investigated the pragmatic and hedonic quality of both interactions \cite{Beurden2012}. Researchers such as Donald Norman, Jakob Nielsen and Malizia et al. point out the importance of gestural guidelines in order to assure usability and the naturalness of gestural interfaces \cite{NormanNielsen2010, Norman2010, Malizia2012}. Additionally, Montero et al. carried out a study on social acceptance of gestural interfaces which describes influencing factors \cite{Montero2010}.
\newline
Gestures in human computer interaction are a new form for input methods and for manipulating interfaces  \cite{Grundmann2016}. However, gestural touch-based interactions are already implemented in many electronic products. Park et al. claimed that those interactions er realistic, natural and intuitive \cite{ParkWonkyu2013}. Nevertheless, defining a set of gestures is essential in order to do the right gesture for a particular action. Two different approaches are suggest by Park et al., a analytically approach, where a selection of gesture is created by expertise an design principle, and an empirical approach which is more user oriented with users defined gestures. Combining the potential of those two approaches where the set of gestures are easy to perform and task oriented, is beneficial \cite{ParkWonkyu2013}. 
\newline
Gestural interfaces have great advantage, especially in situations that require natural manipulation and shortcuts, or even by reducing distraction of visual attention \cite{ParkWonkyu2014}. Furthermore, Riener argued that gestural interactions can minimize visual and cognitive workload and also are easy, enjoyable and encouraging \cite{Riener2012}. Leonardi et al., on the other hand, argued that gestural interfaces have some disadvantages when it comes to precision and navigation in free space \cite{Leonardi2010}. Good eye-hand coordination skills are essential for interacting with gesture-based interface. People with dexterity - or motor performance problems can experience some challenges \cite{Ni2011, Siek2005}. Furthermore, problems can arise when the user gets too relaxed and performance of certain gesture becomes more difficult. Than gestures become to similar during the performance and the system has trouble to distinguish those gestures \cite{Freeman2012}.
\newline

\section{Gesture recognition systems}
\label{sec:handfreesensor}
Recognition of gestures is about analyzing and comparing visual images \cite{Pavlovic1997}. Systems that are specialized in gesture recognition can identify various body and hand postures, convert them into commands which then can be used to interact with computer controlled systems.
There are two different types of gesture recognition systems. Those systems can be categorized in wearable and vision based. A wearable recognition system such as data glove consist of sensors that are mounted on the glove. Those sensors measures hand and finger position in real time and transform the movement into electrical signals. One of the biggest disadvantage of wearables is the acquisition cost and the often are not wireless \cite{Gupta2001}. Visual based systems on the other hand operate more like a scanner, it consist of one or more cameras in order to record hand and body movements \cite{Bilal2013}. A vision based approach is differentiated by 3D and 2D analysis. Unfortunately, creating a computational 3D model of a body part in real time is rather resource intensive \cite{Gupta2001}. Furthermore, gesture recognition system can also be distinguished into user dependent or user independent. A user dependent recognition system has a predefined set of optimized gestures. The advantage of a user dependent system is that a great number of gestures can easily and exact be analyzed and managed. Additionally, gestures can be personalized and are for that reason easier to remember. However, due to the fact that those system focuses only on one user makes it not user-friendly for multiple users. 
Gestures might only be intuitive to one user but not to others. A system that focuses on multiple users benefits on a user independent recognition model where each user can contribute and specify their own gesture to a specific task. Nonetheless, this approach demand a lot of data from each user and the recognition accuracy is quite poor \cite{Kallio2006}.
Other hand pose estimation methods such as  appearance-base and model-based are discussed by Apedan \cite{Abedan2015}. An appearance based estimation method focus on recognition, where a hand pose image is compared to a set of predefined gestures. This approach is best suited for real-time applications where it is focused on recognition of a small amount of gestures. Model-based estimation method on the other hand, provide tracking information during the interaction and calculates global hand motion and joint angels continuously. The high computational complexity makes this approach quite costly and are therefor not quite suitable for real-time applications.


Due to the growing interest in Virtual Reality (VR), several companies have developed various motion controllers such as Eyesight\footnote{http://eyesight-tech.com}, Myo\footnote{https://www.myo.com}, Hapto\footnote{http://hapto.me}, VicoVR\footnote{http://www.vicovr.com}, Nod\footnote{https://nod.com}, Microsoft Kinect\footnote{https://developer.microsoft.com/en-us/windows/kinect} and Leap Motion\footnote{https://www.leapmotion.com}. (Grundmann, 2016)
Each controller has its own advantage and disadvantage. Kinect for instance, is not suitable for recording of small gestures like toe tipping \cite{Galna2014}. It is designed to scan body gestures from a sensor distance of more than 40 cm and it also requires sufficient space around the scanning object \cite{Weinberger2015}.
Leap Motion on the other hand, is specialized in hand and finger tracking. The sensor is design for short distance scanning and therefor does not demand so much space around the object like Kinect does \cite{Grundmann2016}.


\subsection{Leap Motion}
Leap Motion Controller (LMC) is a hardware device which detects finger position, hand position and various gestures. It was invented by the American company, Leap Motion and released in May 2012. In October 2012 a software developer program (SDK) was introduce for those who wanted to develop applications for the controller. In 2016, Orion - a software for Virtual Reality was released. The tiny sensor measures not more than 13mm x 13mm x 76mm and can for instance be connected to a PC via its USB port. The device records and displays live motion picture of one or two hands \cite{Smeragliuolo2016} and can be used for touch-less input commands  \cite{Weichert2013}.
The Leap Motion controller is equipped with two monochromatic infrared cameras that records around 200 frames per second  and three infrared LEDs that produces a pattern-less infrared light. Data from the device is than sent to the Leap Motion software and analyzed. The software compares the 3D position data to the 2D frames \cite{Grundmann2016}.
The controller can be used for many tasks such as navigating, visualization, zooming, etc. Moreover, it also has been used in areas like rehabilitation and it revealed to be a fun an engaging form of therapy \cite{Smeragliuolo2016}.

%18Lu, W., Z. Tong, and J. Chu,2016

\section{Serious games}
\label{sec:seriousgames}
This project is about the assessment of a hands free device and gesture-based computer games as a medical evaluation tool. A medical evaluation tool in form of a computer game is referred to as serious games. Serious games are computer games that are not for entertainment and pleasure but they are specialized for a more serious purpose. 

The concept of serious game started in the early 1970’s. Abt \cite{Abt1970} stated that \textit{“...these games have an explicit and carefully thought-out educational purpose and are not intended to be played primarily for amusement.”}.
Considered as the  world’s first serious video game was launched in the USA in 1972. The game that was a potential educational tool was called Odyssey. Other educational games like “The Oregon Trail” and “Lemonade Stand followed soon after \cite{Laamarti2014}. 
The first simulation tool was launch for the American army in 1981.  The game called “The Bradley Trainer” was developed for training purpose of the Bradly tank.
The market for serious games is a remarkable business and has a enormous growth rate in the lately years. That means that a lot of effort is put into this area and the demand of such systems is significant \cite{Laamarti2014}.
An interesting classification of serious games was suggested by Ratan et al. \cite{Ratan2009}. They classify serious games in four section: (1) educational for academic and social change and health, (2) practicing skills and problem solving, (3) target age group and (4) game platform.
Laamarti et al. also made an attempt to classify serious games. They started with to define the characteristics of serious game that includes activity, modality, interaction style, environment and application area \cite{Laamarti2014}. 

There is a tremendous development of serious games in different areas. Laamarti et al. named domains such as training, education, health care, well-being, advertisement, cultural heritage,  interpersonal communication and others \cite{Laamarti2014}. 
For this project, the area of  bio-medical and health care is of interest and will be explored further. The aim of  serious games in health care is to provide knowledge and skills within the medical domain in order to simulate situations, to guarantee safety, to lower the budget and much more. According to Laamarti et al. \cite{Laamarti2014}, serious games for health care can be classified into health monitoring, detection and treatment, therapeutic education, prevention, and rehabilitation \cite{Laamarti2014}. Due to the fact that a neurological disorder affects many millions of people, rehabilitation of motor skills can have a significant influence on the development of serious games. For recovering from brain lesions a game named The rehabilitation Gaming System (RGS) was designed which can be use either at home or in clinic. Finger, wrist and elbow  movements are recorded with a help of data gloves and video camera \cite{Cameirao2009}.

Concerning factors that makes a serious game successful, the literature showed that researcher point out factors for serious games that motivates, stimulates and pleases the player \cite{Laamarti2014}. To name just a view, these factors are background music for motivating players, providing guidance to prevent confusion, avoiding negativity, visibility of displays, multiplayer collaboration, providing syllabus for educational purpose, provide an appropriate challenge for each level.
Laamarti el al. \cite{Laamarti2014}, tok those factors in consideration when they proposed guidelines regarding the design and development of serious games. They suggested that row of design elements such as software should be user-centered, multimodal, invigorate virtual connectedness, the possibility to adapt to players, standard evaluation i order to gain higher credibility and sensory- based stimulation \cite{Laamarti2014}.
Interestingly, they pointed out the need of focusing more on natural interfaces in serious game design. Additionally, the success of a  serious games is characterized by terms like  pleasurable and engaging although the game has a serious goal. However, a good balance between pleasure and purpose is essential when designing serious games.
Other important elements that needs to be taken in consideration when dealing with design of serious games are fantasy, fidelity, and context
\cite{Charsky2010}. These three elements are of great value because they provide authenticity, assist in recognizing and understanding demanding context.
Tong et al. (2016) did a research on serious game based screening tool in order to see if it is possible to use such games as a tool of prediction for delirium. They believe that serious games are promising for cognitive screening in clinical settings.  Based on their research they recommend multiple gesture to maximize interaction, incorporate validate psychological task , the game should be enjoyable and easy, there should be multiple version of the game in order to adjust the needs of diverse target group \cite{Tong2016}. 
Another research on serious game as screening method was made by Boletsis et al.\cite{Boletsis2016}. They developed Smartkuber, a serious game for cognitive screening of elderly players. The interaction tool for this game was based on Augmented Reality (AR). They pointed out the disadvantage such as psychological stress related bias, improved performance through the learning effect, lack of motivation and economic burden, of nowadays standard pen-paper and computational test. On the other hand, the advantage of cognitive screening serious games as a screening tool can be significant as they save time and cost, are more accurate, reduce psychological stress, are enjoyable and self-administered, just to name a few.  Though the playful interaction and stimulating exercise the players cognitive ability was screened and monitored on a frequent basis. The obtained result was more objective since the game was played on the players premise, this means the players decided time and place of their choice. They minimized the learning effect through randomized mini games \cite{Boletsis2016}. 
 


The literature review showed that there is a few research on serious games in the area of medical research and screening \cite{Tong2016, Ben-Sadoun2018}. Unfortunately, the literature review provided very little relevant information within ASD and gesture hand free serious games which could have been useful for this study.



\section{Design principles for gesture interfaces and design criteria for serious games}
\label{sec:designprinciples}
Donald Norman created design principles as a guide when designing everyday things. He stated that that a conceptual model is important to manage everyday thing, because humans need to foresee the consequence of their behaviour. Otherwise, people do things by habitual repetition and get in trouble if something goes wrong because they do not know how things are interrelated \cite{Norman2002}. Another issue that Norman pointed out is visibility. The danger with invisible function is that user can forget them, the system gets less understandable and the operations become more difficult. Norman also mentioned the principle of feedback, an essential point for science of control and information theory. Feedback is about notifying the user who just has performed an action. Ideally, if a human is able to mentally perceive the meaning and understands how to interact with a system, then the principal of affordance is achieved. The lack of these principles is something that Nielsen pointed out when he tested the Kinects gestural interfaces in 2010 \cite{Nielsen2010}. To make gestural interaction with touch-less interfaces user friendly is important for it success in the future. 
Gestural touch-less interface is a great idea. However, when creating such interfaces it is important to understand the challenges it arises. Gestural touch-less interface must be designed universal and provide the user with good user experience. It is essential that every user understands the system and knows how to control it. Otherwise, the user gets frustrated and discards the system. 
\\
%\newline
When designing games for health care, rehabilitation or preventative measure, it is essential to have in mind the cognitive and motor ability which can vary from user to user \cite{Mendes2012}. Games must be design to meet the right target group and follow design guidelines in order make it fun and motivational for the user \cite{Mendes2012}. Keywords such as adaption, personalisation and calibration should be outline the guidelines for serious games \cite{Goebel2010}. Automatic calibration of difficulty levels and dynamic adaptability are important because individuals have their own pace and endurance especially if they are cognitive or motor impaired \cite{Geurts2011, Bouchard}. Furthermore, the age aspect of the target group also need to be taken in consideration. Designing for elderly user demands different criteria than designing for toddler. For instance, an elderly user will need bigger letters and bigger buttons \cite{Palacio2012}. 
Games for rehabilitation  should contain features such as accuracy in order to analyze and evaluate users performance, low-cost and portable, real-time feedback to the clinician, customized and adaptable i relation to ability, calibration of the system in relation the players ability, feedback and rewarding for motivation and fun \cite{Lewis2012}.

Moreover, as a result of their own observation and experience, Paraskevopoulos et al. complemented the previous mentioned guidelines. They discovered that games based on sport, hobby or other known activities are a great inspiration source for users. Furthermore, an adjustable level of challenge is important in order to offer the user a good and challenging experience. Additionally, stimuli is an important feature to enhance enjoyment and engagement. Stimuli can be either visual in form of great graphic and figures or it can be an motivational and cheerful audio feedback \cite{Paraskevopoulos2014}. Moreover, clear instructions and a description of the games intention and its target encourages the players and provides cognitive aid and by that decrease the cognitive load \cite{Paraskevopoulos2014, Koridon2016}. Lastly, games that evoke competitive feelings are motivational and encourage the player to carry on \cite{Paraskevopoulos2014}.




\section{Developmental screening method}
\label{sec:screeningmethod}
Nowadays,checking children for developmental deficit are a part of the routine examination in Germany. In child care centers the children are extensively observed, once a year.
A more detailed examination which is a development screening procedure is carried out at the pediatrician. A so-called "U-Untersuchung" is a non-obligatory examination to detect early signs of development deficit and other diseases \cite{kindergesundheitinfo, Uuntersuchung}.
An additional examination is performed by the educational authority in order to check if the child is ready for school.
In Norway, there are similar routines where childrens are observed and evaluated in relation the the national development plan.

However, this study concerns the procedure of motor skills test methods. A well known method is described in the subsection below. 

%https://www.familie-und-tipps.de/Gesundheit/Kinderkrankheiten/U-Untersuchungen/

%https://www.kindergesundheit-info.de/themen/entwicklung/frueherkennung-u1-u9-und-j1/


\subsection{The Movement Assessment Battery}
There are several clinical instruments for testing children on motor performance and other capabilities. The Movement Assessment Battery - second edition (M-ABC2) is a well known test battery for motor-coordination capability of children from three to 16 years of age. It provides a standard performing test and checklist and is designed to identify motor coordination impairments. However, like other test methods, M-ABC2 has its considerable strength and weakness. For instance, Ted Brown \cite{Brown2009} pointed out the lack of reliability and validity of this test.

Dr. phil. Thorsten Macha provides a useful overview of several test methods and procedures including M-ABC2. He describes specific motor exercises to test manual dexterity, ball skills and balance \cite{Macha2010}.

Exercises suggested for the M-ABC2 test were the point of departure for this project and the possibility of digital implementation was explored.



%http://entwicklungsdiagnostik.de/motoriktests.html


%https://www.tandfonline.com/doi/abs/10.1080/01942630802574908?src=recsys&journalCode=ipop20



\section{Autistic Spectrum Disorder (ASD)}
\label{sec:asd}

Autism Spectrum Disorder (ASD) is a medical term for a collection of different autistic disorders which includes autistic disorder, Asperger disorder and pervasive developmental disorder. These disorders are often practically hard to distinguish and for that reason a medical term ASD was established.  ASD is a complex neurological developmental disorder \cite{BundesverbandAutismusDeutschland}. The onset of the symptoms are recognized early in the childhood \cite{NationalInstituteofMentalHealth}. The development delay and impairment of a child affects the social behaviour, they have difficulties to express themselves and find it hard to communicate with others.Furthermore, children with ASD have often motor abnormal behavior such as repetitive and stereotyped movement \cite{McCleery2013,Lloyd2013}.  ASD is therefore characterized as a social and cognitive disorder \cite{Whyatt2013}. The impairment ranges from mildly to severity and varies from person to person \cite{NationalInstituteofMentalHealth}. However, ASD is more common in males than in females \cite{Ecker2017}. According to Bundesverband for Autismus, studies in Europe, Canada and USA discovered that 6-7 per 1000 people have ASD \cite{BundesverbandAutismusDeutschland}. 



\subsection{Motor skills associated with ASD}

It is often observed motor delays and difficulties with gross and fine motor coordination in connection with ASD \cite{McCleery2013}. This is also confirmed in several studies for instance by Founier et al. \cite{Fournier2010}. Motor delays gets more obvious with age and motor skill fall behind the expected normal chronological age \cite{Lloyd2013}. motor skill are complicated movements that needs coordination and motor planning skills \cite{Lloyd2013}. 
Also imitation problems were found in the research done by Siek et al. \cite{Siek2005}
Some researcher refer to motor clumsiness as a clinical description of the motor skill disorder \cite{Lloyd2013}. %kanner
One of the reason why children with ASD have social, emotional and communicative difficulties is the motor control deficit and eye-hand coordination inefficiency \cite{Crippa2013}. Motor control is important in order to express emotions, social engagement and it corroborates cognitive development \cite{Anzulewicz2016}. 
Disturbance in motor movements like to gasp, touch, handwriting, body-posture shift and walk are frequently seen in children with ASD. Additionally, significant motor coordination and motor timing deficit in individuals suffering from ASD is revealed by several research \cite{Anzulewicz2016}.

Research of motor movements in game-play on a tablets revealed that children with autism used greater force of impact and gesture pressure, a different distribution of force on the devise and faster gestures during the game-play \cite{Anzulewicz2016}. This characterizes the motor control deficit on children with ASD. However,  Fast hand movements can be an explanation of eye-hand coordination problems. Crippa pointed out that the reaction time of ASD patients slowed down  when a higher degree of eye-hand coordination was needed \cite{Crippa2013}.




\subsection{Diagnosing and screening methods}

Screening is a standardized and systematic process in order to detect early sign of a disability or disease \cite{screening}. A screening of young children would ensure that early signs of ASD are detected and diagnosed in an early stage of their live in order to enhance the chance of a better live \cite{Zwaigenbaum2015}.
However, it is quite challenging to discover ASD because the diagnosis of autism is complex. It relies on specialist and their diagnostic instruments and how they interpret the data that is gained during the test sessions, observations and interviews with the parents \cite{Anzulewicz2016}.


In the area of clinical assessment of motor function, test like M-ABC and Mullen Scale are used frequently. 
Lloyd et al. used the Mullen Scale of Early Learning (MSEL) development test for children from 0 to 68 month in his study on motor skills of toddlers with ASD. It is divided into 5 categories,: gross motor, fine motor, visual reception,  receptive language , expressive language \cite{Lloyd2013}.
M-ABC is movement assessment battery where children are divided in three age groups. This test focuses on  motor coordination development.  
Other evaluation test for  motor skills may such as  Pearbody Developement and Motor Scale - 2 might be more suitable according to Ozonoff et al. \cite{Ozonoff2008}. 
He also use the Vineland Adaptie Behaviour scale (VABS) which is a standardized parent report. It measures the ability of everyday functioning of the child. Patent Surveys is a cheap method but lacks of precise interpretation.

Systems such as optical motion tracking on the other hand are an expensive laboratory based system that requires expertise and are more ornate and costly \cite{Anzulewicz2016}. 

Anzulewicz et al. pointed out that more accessible and precise computations system are needed to measure motor performance \cite{Anzulewicz2016}.


According to Norsk Nevrologisk forening \cite{nevrolegehandboka2016}, the Autism Spectrum Quotient (AQ) is a wide used screening tool for initial diagnostic of ASD in Norway. The test consist of a report with 50 questions and a score higher than 32 could indicate Asperger syndrome (AS) .

However, a screening process is often related to time consumption and reimbursement. Zwaigenbaum et al. \cite{Zwaigenbaum2015} discovered during their literature review that pediatricians complain about insufficient time and compensation and stated this as the biggest hindrance for conducting a screening. Furthermore, other issues concerned the distribution of screenings questionnaires, interrupted work-flow, scoring difficulties and the lack of training regarding instruments for  screening of ASD. In order to gain more acceptance and a regular screening routine, Zwaigenbaum et al., believe that a  screening of multiple disabilities at the same time could be of great interest.






