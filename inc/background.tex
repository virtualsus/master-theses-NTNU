\chapter{Background, theory and existing literature}
\label{chap:background}


\section{Keywords}
\label{sec:keywords}

\section{Gesture definition and guidelines}
\label{sec:gesture}
Gestures are very common in intercommunications between humans. Arm, face and head movements are gestures that are evoked by an idea, emotion or reaction. For a better understanding, gestures accompany verbal information. However in some situations are gestures more suitable than speech, such as in noisy environments (Wagner et al., 2014).
Some HCI researcher classify and categorize gesture in order to differentiate them from each other. McNeill (2005) for instance, categorized gestures into gesticulation, pantomime, emblem, sign language. He then classified gesticulation into iconic, metaphoric, rhythmic, cohesive, and deictic (McNeill, 2005). Pavlovic et al. subdivides gesture into communicative (gestures that have to be learned) and natural hand and arm movements, called manipulative gestures (Pavlovic et al., 1997). Park and Han distinguished between manipulative and communicative gestures, which are a part of a nonverbal component of speech (Park and Han, 2013). Furthermore, an interesting claim made Riener. He pointed out that gestures might differ from gender, ethnicity and cultural background (Riener, 2012).
 
Gesture Guidelines
Gestures can be part of a multimodal application, which can be a solution for people with challenges in everyday life in order to compensate impairments such as speech or hearing disability. Anastasiou (2012) carried out a study where a subject navigated a wheelchair through a set-up of smart home environment by only using speech and gestures. Her study showed that the subjects mostly used gesture when something went wrong (Anastasiou, 2012). Another study, relevant for this project was conduct by van Beurden et al. (2012) which compared gestured-base investigation to device-based interaction  and investigated the pragmatic and hedonic quality of both interactions (Van Beurden et al., 2012). Researchers such as Donald Norman, Jakob Nielsen and Malixia et al. point out the importance of gestural guidelines in order to assure usability and the naturalness of gestural interfaces.(Norman and Nielsen, 2010, Norman, 2010, Malizia and Bellucci, 2012). Additionally, Montero et al. carried out a study on social acceptance of gestural interfaces which describes influencing factors (Montero et al., 2010).



\section{Autistic Spectrum Disorder (ASD)}
\label{sec:asd}

Autism Spectrum Disorder (ASD) is a medical term for a collection of different autistic disorders. (autismus.de ) which includes autistic disorder, asperger disorder and pervasive developmental disorder. These disorders are often practically hard to distinguish and for that reason a medical term ASD was established.  ASD is a complex neurological developmental disorder. The onset of the symptoms are recognized early in the childhood. (NIHM). The development delay and impairment of a child affects the social behaviour, they have difficulties to express semselv and find it hard to communicate with others.Furthermore, children with ASD have often motor abnormal behavior such as repetitive and stereotyped movement (McCleery. lloyd).  ASD is therefore characterized as a social and cognitive disorder.(whyatt) The impairment ranges from mildly to severity and varies from person to person. (NIHM) however, ASD is more common in males than in females (eckert and others). According to autisme.de, studies in Europe, Canada and USA discovered that 6-7 per 1000 people have ASD. 



\subsection{Motor skills associated with ASD}

It is often observed motor delays and difficulties with gross and fine motor coodination ( McCleery) in connection with ASD.  This is also confirmed in serveral studies for instance by Founier et al, Motor delays gets more obvious with age and motor skill fall behind the expected normal chronological age. (lloyd) motor skill are complicated movements that needs coordination and motor planning skills. 
Also imitation problems were found in the research done by  Roger et al
Some researcher refer to motor clumsiness (kanner, Lloyd) as a clinical description of the motor skill disorder.
One of the reason why children with ASD have social, emotional and commicative difficulties is the motor control deficit and eye-hand coordination ineffiency. (crippa). Motor control is important in order to express emotions, social engagement and it corroborates cognitive development. (anzulewixc,) 
Disturbance in motor movements like to gasp, touch, handwriting, bodyposture shift and walk are frequently seen in children with ASD. (azulewicz) Additionally, significant motor coordination and motor timing deficit in individuals suffering from ASD is revealed by several research. (azulwicz)

Research of motor movements in gameplay on a tablets showed that children with autisme used greater force of impact and gesture pressure, a different distribution of force on the devise and faster gestures during the gameplay (anzulewicz). This caracaterixes the motor control deficit on chidlren with ASD. However,  Fast hand movements can be an explanation of eye-hand coordination problems (crippa) Crippa pointed out that the reaction time of ASD patients slowed down  when a higher degree of eye-hand coordination was needed.




\subsection{Diagnosing and screening methods}

Screening is a standardized and systematic process in order to detect early sign of a disability or disease. A screening of young children would ensure that early signs of ASD are detected and diagnosed in an early stage of their live in order to enhance the chance of a better live. (Zwaigenbaum et al. 2015)  
However, it is quite challenging to discover ASD because the diagnosis of autism is complex (Anzulewicz). It relies on specialist and their diagnostic instructuments and how they interpret the data that is gained during the test sessions, observations and interviews with the parents.


In the area of clinical assessment of motor function, test like M-ABC and mullen Scale are used frequently. [Reference]
Lloyd et al used the Mullen Scale of Early Learning ( MSEL) development test for children from 0 to 68 month in his study on motor skills of toddlers with ASD. It is divided into 5 categories,: gross motor, fine motor, visual reception,  receptive language , expressive language.
M-ABC is movement assessment battery where children are divided in 3 age groups. This test focuses on  motor coordination development.  
Other evaluation test for  motor skills may such as  Pearbody Developement and Motor Scale - 2 might be more suitable according to Ozonof et al 2008 (Lioyd))
He also use the Vineland Adaptie Behaviour scale (VABS) which is a standardized parent report. It measures the ability of everydays functioning of the child. Patent Surveys is a cheap method but lacks of precise interpretation.

Systems such as optical motion tracking on the other hand  are an expensive laboratory base system that requires expertise and are more ornate and costly. [Reference]

Anzulewiz et al. pointed out that more accessible and precise computations  system are needed to measure motor performance.


According to nevro.legehandboka.no (http://nevro.legehandboka.no/handboken/sykdommer/alle-sykdommer/alfabetisk-oversikt/autisme-aspergers-syndrom), the Autism Spectrum Quotient (AQ) is a wide used screening tool for initial diagnostic of ASD in Norway. The test consist of a report with 50 questions and a score higher than 32 could indicate Asperger syndrome (AS) .

However, a screening process is often related to time consumption and reimbursement. Zwaigenbaum et al discovered during their literature review that pediatricians complain about insufficient time and compensation and stated this as the biggest hindrance for conducting a screening. Furthermore, other issues concerned the distribution of screenings questionnaires, interrupted work-flow, scoring difficulties and the lack of training regarding instruments for  screening of ASD. In order to gain more acceptance and a regular screening routine, Zwaigenbaum et al, believe that a  screening of multiple disabilities at the same time could be of great interest.



\section{Developmental screening method}
\label{sec:screeningmethod}
Nowadays,checking children for developmental deficit are a part of the routine examination in Germany. In child care centers the children are extensively observed, once a year.
A more detailed examination which is a development screening procedure is carried out at the pediatrician. A so-called "U-Untersuchung" is a non-obligatory examination to detect early signs of development deficit and other diseases.
An additional examination is performed by the educational authority in order to check if the child is ready for school.
In Norway, there are similar routines where childrens are observed and evaluated in relation the the national development plan.

However, this study concerns the procedure of motor skills test methods. A well known method is described in the subsection below. 

%https://www.familie-und-tipps.de/Gesundheit/Kinderkrankheiten/U-Untersuchungen/

%https://www.kindergesundheit-info.de/themen/entwicklung/frueherkennung-u1-u9-und-j1/


\subsection{The Movement Assessment Battery}
There are several clinical instruments for testing children on motor performance and other capabilities. The Movement Assessment Battery - second edition (M-ABC2) is a well known test battery for motor-coordination capability of children from 3 to 16 years of age. It provides a standard performing test and checklist and is designed to identify motor coordination impairments. However, like other test methods, M-ABC2 has its considerable strength and weakness. For instance, Ted Brown [reference] pointed out the lack of reliability and validity of this test.

Dr. phil. Thorsten Macha provides a useful overview of several test methods and procedures including M-ABC2. He describes specific motor exercises to test manual dexterity, ball skills and balance. 

Exercises suggested for the M-ABC2 test were the point of departure for this project and the possibility of digital implementation was explored.



%http://entwicklungsdiagnostik.de/motoriktests.html


%https://www.tandfonline.com/doi/abs/10.1080/01942630802574908?src=recsys&journalCode=ipop20

\section{Leap Motion}
\label{sec:leapmotion}
\section{Serious games}
\label{sec:seriousgames}
This project is about the assessment of a hands free device and gesture-based computer games as a medical evaluation tool. A medical evaluation tool in form of a computer game is referred to as serious games. Serious games are computer games that are not for entertainment and pleasure but they are specialized for a more serious purpose. 

The concept of serious game startet in the early 1970’s. Abt (reference) stated that “...these games have an explicit and carefully thought-out educational purpose and are not intended to be played primarily for amusement.”
Considered as the  world’s first serious video game was launched in the USA in 1972. The game that was a potential educational tool was called Odyssey. Other educational games like “The Oregon Trail” and “Lemonade Stand followed soon after. (Fedwa)
The first simulation tool was launch for the American armee in 1981.  The game called “The Bradley Trainer” was developed for training purpose of the Bradly tank.
The market for serious games is a remarkable business and has a enormous growth rate in the lately years. That means that a lot of effort is put into this area and the demand of such systems is significant.(Fedwa)
An interesting classification of serious games was suggested by Ratan et al (Refernce). They classify serious games in four section: [1] educational for academic and social change and health, [2] practicing skills and problem solving, [3] target age group and [4] game platform.
Laamarti et al (2014) also made an attempt to classify serious games. They started with to define the characteristics of serious game that includes activity, modality, interaction style, environment and application area. 

There is a tremendous development of serious games in different areas. Laamarti et al (2014) named domains such as training, education, health care, well-being, advertisement, cultural heritage,  interpersonal communication and others. 
For this project, the area of  biomedical and health care is of interest and will be explored further. The aim of  serious games in health care is to provide knowledge and skills within the medical domain in order to simulate situations, to guarantee safety, to lower the budget and much more. According to Laamarti et al (2014),  serious games for health care can be classified into health monitoring, detection and treatment, therapeutic education, prevention, and rehabilitation. Due to the fact that a neurological disorder affects many millions of people,  rehabilitation of motor skills can have a significant influence on the development of serious games. For recovering from brain lesions a game named The rehabilitation Gaming System (RGS) was designed which can be use either at home or in clinic. Finger, wrist and elbow  movements are recorded with a help of data gloves and video camera.  [Reference: Cmeirao]

Concerning factors that makes a serious game successful, the literature showed that researcher point out factors for serious games that motivates, stimulates and pleases the player. [Reference Laamarti]To name just a view, these factors are background music for motivating players, providing guidance to prevent confusion, avoiding negativity, visibility of displays, multiplayer collaboration, providing syllabus for educational purpose, provide an appropriate challenge for each level.
Laamarti el al (2014), tok those factors in consideration when they proposed guidelines regarding the design and development of serious games. They suggested that row of design elements such as software should be user-centered, multimodal, invigorate virtual connectedness, the possibility to adapt to players, standard evaluation i order to gain higher credibility and sensory- based stimulation.
Interestingly, they pointed out the need of focusing more on natural interfaces in serious game design. Additionally, the success of a  serious games is characterized by terms like  pleasurable and engaging although the game has a serious goal. However, a good balance between pleasure and purpose is essential when designing serious games.
Other important elements that needs to be taken in consideration when dealing with design of serious games are fantasy, fidelity, and context. (Charsky (2010)).  These three elements are of great value because they provide authenticity, assist in recognizing and understanding demanding context.
Tong et al (2016) did a research on serious game based screening tool in order to see if it is possible to use such games as a tool of prediction for delirium. They believe that serious games are promising for cognitive screening in clinical settings.  Based on their research they recommend multiple gesture to maximize interaction, incorporate validate psychological task , the game should be joyable and easy, there should be multiple version of the game in order to adjust the needs of diverse target group. 
Another research on serious game as screening method  was made by Boletsis et al. They developed Smartkuber, a serious game for cognitive screening of elderly players. The interaction tool for this game was based on Augmented Reality (AR). They pointed out the disadvantage such as psychological stress related bias, improved performance through the learning effect, lack of motivation and economic burden, of nowadays standard pen-paper and computational test. On the other hand, the advantage of cognitive screening serious games as a screening tool can be significant as they save time and cost, are more accurate, reduce psychological stress, are joyable and self-administered, just to name a few.  Though the gameful interaction and stimulating exercise the players cognitive ability was screened and monitored on a frequent basis. The obtained result was more objective since the game was played on the players premise, this means the players decided time and place of their choice. They minimized the learning effect through randomized minigames. 
 


The literature review showed that there is a few research on serious games in the area of medical research and screening. (Tong, Ben-Sadoun) Unfortunately, the literature review provided very little relevant information within ASD and gesture hand free serious games which could be useful for this study.



\section{Design principles for gesture interfaces and design criteria for SG}
\label{sec:designprinciples}
Donald Norman created design principles as a guide when designing everyday things. He stated that that a conceptual model is important to manage everyday thing, because humans need to foresee the consequence of their behaviour. Otherwise, people do things by habitual repetition and get in trouble if something goes wrong because they do not know how things are interrelated (Norman, 2002). Another issue that Norman pointed out is visibility. The danger with invisible function is that user can forget them, the system gets less understandable and the operations become more difficult. Norman also mentioned the principle of feedback, an essential point for science of control and information theory. Feedback is about notifying the user who just has performed an action. Ideally, if a human is able to mentally perceive the meaning and understands how to interact with a system, then the principal of affordance is achieved. The lack of these principles is something that Nielsen pointed out when he tested the Kinects gestural interfaces in 2010 (Nielsen, 2010).  To make gestural interaction with touchless interfaces user friendly is important for it success in the future. The question is if Normans design principles are outdated or still valid for gestural interface and can these be the point of departure for touchless interface design or must are new principles needed.
Gestural touchless interface is a great idea. However, when creating such interfaces it is important to understand the challenges it arises. Gestural touchless interface must be designed universal and provide the user with good user experience. It is essential that every user understands the system and knows how to control it. Otherwise, the user gets frustrated and discards the system. For that reason, it is important to investigate what effect the gestural interfaces have on the use, how user responds to gestured controlled interfaces. The outcome then points out difficulties and challenges which is of great value for designers and technologist.

Design criterias for serious games


\subsection{ISO standard}
\label{sec:isostandard}


