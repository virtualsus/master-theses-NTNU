\chapter{Conclusion}
\label{chap:conclusion}

%used as a complementary /supplementary tool 

%Proof of concept (PoC) is a realization of a certain method or idea in order to demonstrate its feasibility

%Re-introduce the project and the need for the work
%Re-iterate the purpose and specific objectives of your project
%Re-cap the approach taken
%Summarize the major findings
%recommendations of your work

Free-hand gesture interaction with computer system and interfaces is a relative new modality. This field is of growing interest not only because it is an enjoyable kind of interaction but it is also important when visual focus is needed. Some researcher argue that free-hand gesture interaction is convenient for manipulating systems because gestures are natural to human. Gestures are used in humans daily behaviour, for instance by saying goodbye, ok or pointing at something. 
Free-hand gesture interfaces are already employed in industry and in health care. However, using free-hand gesture in a screening type of method is still little explored.

The concept of this study was to examine an idea about using Leap Motion and free-hand gesture interaction in order to measure motor performance in children. This was a prove of principle kind of approach where the specific idea was checked for its feasibility. Thus, the goal was to investigate the feasibility of using Leap Motion controller to extract motor performance parameters. Since there is only few research in this field, the value of project was significant.

In order to prove the concept and investigate possibilities, a simple prototype was created in Processing, an open source computer language. This program was established mainly for electronic art and visual design but it is suitable for prototyping as well. Processing was a convenient prototype tool for this project since it is build on Java programming language and most important because of its compatibility to Leap Motion. Using Leap motion as a motion sensor tool was well-chosen due it economical benefits and the fact that it could record finger and hand gestures.\\
At the beginning of the project, interviews in form of a brief conversation were carried out to ascertain existing screening method for motor performance and the fact that it could be used to determine Autism. Autism was pointed out as an example to illustrate what kind of disorder could be detected by applying the free-hand gesture concept. The inspiration for the concept was based on an existing motor coordination test called Movement Assessment Battery for Children (MABC-2) which is a standard performance test that includes specific motor exercises.  
Ideas from the MABC-2 were used to create a concept that included a collection of five mini games. However, to implement and investigate these five games would be beyond the scope of this project. Thus, only one games was selected and chosen to be implemented further. "Mole in the hole" is a free-hand gesture based game where the player must guide the mole through its tunnel on the way out. During the game parameters such as time and accuracy were recorded. The goal of the game was to move the mole as fast as possible and to be accurate as much as possible.
The game was tested on 8 children, the youngest child was 3 years of age and the oldest child was 13 years of age. The test showed that children under 5 did not understand the free-hand concept. They constantly touched the screen in order to move the mole. Moreover, the sensor had problems to detect gestures from toddlers due to the tiny size of the finger and the vague gestures. Consequently, this method is not applicable on children younger than 5 years old. 
Furthermore, observation revealed that all children showed interest in the game and enjoyed the free-hand concept which could be explained by the fact that none of them had pre-experience in free-hand gesture interactions. It seemed that an appealing interface has a great impact on how children react to serious games like suggested in this project. The children were pointing out the mole figure as the most liked element in the game. 
%sag noch was zu ASD

To sum up, the study demonstrated that it is possible to extract motor performance parameters, such as time and accuracy, from a Leap Motion driven computer game. However, it is not possible to apply these kind of screening method on toddlers since their motor performance skills are not enough developed and they do not have a conceptual understanding of the free-hand concept. Additionally, tiny fingers and vague gesture were the reason why the sensor had problems to detect gestures from toddlers.
Moreover, using the freehand gesture concept as a stand alone screening method is not advisable. As the interview with the pediatrician confirmed, it should be used as a supplementary tool in a more complex screening procedure. Due to the fact that this method provides objective data, it could be a valuable tool to determine developmental disorder.  



\section{Future Work}
\label{sec:future}
%Make recommendations for future research
%test mot autisme
%test med flere personer
%lage flere tester
%Base on the experience of this study , the following recommendation can be presented
%what is left out, yet to-do
%Due to the limited time scope of this study, many interesting points needed to be left out.


The information and experience gained during this study is a great point of departure for further exploration and research.  
Firstly, it is recommended to improve the prototype and to do extensive testing on more children with a broader age rage and a more spread sample. Secondly, it can be a good idea to implement the other games suggested in this study in order to provide varied test methods.
Moreover, further investigation whether this is a suitable screening method for determining Autism in children would have been beneficial. This can be done for instance by sampling a two test groups. One group samples children with Autism and the other group non-autistic children. The result of those group could than have been compared and analyzed. 
Beside testing this concept for screening method it could be also interesting to explore whether free-hand gesture interactions is suitable as a therapy tool for neurological injuries. That could be either in form of a PC game or combined with Arduino. The idea is to navigate an Auduino car with specific gestures in order to enhance hand mobility.  


%test for theray tool
%test with arduino
%make a better prototype and test more children
%test with children som har et motor problem


%first of all