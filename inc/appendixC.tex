\chapter{Appendix - Processing Code}

\lstset{language=Java}
\begin{lstlisting}[caption = {}, label={lst:Java}]
import com.leapmotion.leap.*;
Controller leap;
PImage backgroundImage;
PImage MoleState1;
PImage MoleState2;
PImage MoleJump;
PImage MoleGo;
PImage MoleKaBoom;
PImage Timer;
PShape pointer;
int cols, rows;
int w = 80;
int [] grid; //one dimension array
int windowWidth = 1200;
int windowHeight = 800;
int state = 0;
int [][] gridArray = new int [15][10];
int countOK = 0;
int countKaBoom = 0;
int stateEnd = 0;
int counterState = 0; //status for timer
int counterTouching =0;

void setup()
{
  size(1200, 800);//(1440, 960);
  
  backgroundImage = loadImage("Mole-in-the-hole-800x1200-GRIDinBack.jpg");
  backgroundImage.resize(1200, 800);
  MoleState1= loadImage("MoleState1.png");
  MoleState2= loadImage("MoleState2.png");
  MoleJump = loadImage("MoleJump.png");
  MoleGo = loadImage("MoleGo.png");
  MoleKaBoom = loadImage("MoleKaBoom.png");
  Timer = loadImage("timer.png");

  pointer = createShape(ELLIPSE, 0, 0, 40, 40);
  pointer.setFill(color(127, 0, 0));

  leap = new Controller();
}
void draw()
{
  background(backgroundImage);

  //KaBoom counter & timer
  CounterKaBoom();
  Timer(0);// state 0 , not startet
  image(MoleKaBoom, 30, 50);
  image(Timer, 30, 150);

  InteractionBox iBox = leap.frame().interactionBox();
  Pointable pointable = leap.frame().pointables().frontmost();

  Vector normalizedPosition = iBox.normalizePoint(pointable.stabilizedTipPosition());
  float pixelX = normalizedPosition.getX() * windowWidth;
  float pixelY = windowHeight - normalizedPosition.getY() * windowHeight;
  int cx = (int) pixelX - 10;
  int cy = (int) pixelY - 100;

  cols = width/w;
  rows = height/w;
  
  Finger finger = new Finger(pointable);
  if(finger.isExtended())
  {
    Timer(1);
  }
  
  //creates grid
  for (int j = 0; j< rows; j++)
  {
    for (int i =0; i< cols; i++)
    {
      int x = i*w;
      int y = j*w;


      Pointable.Zone fingerzone = pointable.touchZone();

      if (state ==0) //state 0 means , do the first block of code - move the pointer to maul and poke him
      {
       if (counterTouching < 5000)//dersom touches are less than x, repeat the following code
        {
         
          shape(pointer, cx, cy);
          
          // println("State3: " + state);   
          if (fingerzone != Pointable.Zone.ZONE_TOUCHING )//start this code if the finger is at 80/480 ((x == 80 && y==480 &&)
          {
            
            show(MoleState1);
          }
          if (cx >=80 && cx <=160 && cy >= 480 && cy <= 560)
          {
            // image(MoleState2, 80, 480);
            switch (pointable.touchZone()) {
            case ZONE_NONE:
              //Handle distant pointable
              image(MoleState1, 80, 480);
              break;
            case ZONE_HOVERING:
              //Handle pointable near touch plane
              hide(MoleState1);
              image(MoleState2, 80, 480);
              break;
            case ZONE_TOUCHING:
              //Handle pointable penetrating touch plane
              //println(count);
              counterTouching = counterTouching + 1;
              image(MoleJump, 80, 480); //mole pic nr 3
               println("touching: " + counterTouching);
             // println("State3: " + state);
              break;
            default:
              //Handle error cases...
              break;
            }//end switch
          }
        
        }
        else//counterTouching is 2 or more
        {
            state = 1;

        }
      }//end if state

    }
  }
  if (state == 1)
  {

    gridArray[1][6] = 1;//start cell
    gridArray[1][5]= 1;//border cell from start cell
    gridArray[1][7]= 1;//border cell from start cell
    gridArray[2][6] = 1;
    gridArray[3][6] = 1;
    gridArray[3][7] = 1;
    gridArray[4][7] = 1;
    gridArray[5][7] = 1;
    gridArray[5][6] = 1;
    gridArray[5][5] = 1;
    gridArray[5][4] = 1;
    gridArray[6][4] = 1;
    gridArray[7][4] = 1;
    gridArray[7][5] = 1;
    gridArray[7][6] = 1;
    gridArray[8][6] = 1;
    gridArray[8][7] = 1;
    gridArray[9][7] = 1;
    gridArray[10][7] = 1;
    gridArray[10][6] = 1;
    gridArray[11][6] = 1;
    gridArray[11][5] = 1;
    gridArray[11][4] = 1;
    gridArray[10][4] = 1;
    gridArray[4][9] = 1;
    gridArray[9][4] = 1;
    gridArray[9][3] = 1;
    gridArray[9][2] = 2;//end Cell

    int coordX = Math.max(Math.min(cx/80, 14), 0);//fixes out of arrayoutofboundary
    int coordY = Math.max(Math.min(cy/80, 9), 0);


    if (coordX == 9 && coordY == 2) // End cell
    {
      countOK++;
      println("TEST"+ coordX + " " +coordY);
      println("HURRRA");
      stateEnd = 1;
      setHide(true, MoleGo);
      image(MoleJump, 720, 160);
      noLoop();
    } else if (gridArray[coordX][coordY] == 1) // Legal
    {
      countOK++;
     // println("OK:" + countOK);
      
      image(MoleGo, cx, cy);
    } else // Illegal
    {
      countKaBoom++;
     // println("KaBOOM:" + countKaBoom);
       image(MoleGo, cx, cy);
    }

    if (stateEnd == 1)
    {
       fill(255, 255, 0);
       text("HURRA!", 600, 200); 
      textColor1();
    }
  }//end state 1
}//end draw


void show(PImage img)
{
  setHide(false, img);
}
void hide(PImage img)
{
  setHide(true, img);
}
void setHide(boolean hideMole, PImage img)
{
  if (hideMole == false && img == MoleState1 )
  {
    image(MoleState1, 80, 480);
  } 
  if (hideMole == false && img == MoleState2 )
  {
    image(MoleState2, 80, 480);
  } 
  if (hideMole == false && img == MoleGo )
  {
    image(MoleGo, 720, 240);
  } else
  {
    hideMole = true;
    //need to hide the mole behind the background.
  }
}

void textColor1()
{
  textSize(100);
  fill(255, 255, 0);
  text("HURRA!", 600, 160); //what is a proper feedback for a child? picture/text
}
void CounterKaBoom()
{

  text("Dunk: " + countKaBoom, 110, 120);
  fill(255, 255, 255);
  textSize(20);
}

void Timer(int startTimer)
{

  if (startTimer == 1)
  {
    int currentMillis = millis();
    setSeconds(currentMillis);
    setMinutes(currentMillis);
  }
  text("Løpetid: " +  getMinutes() +":" +  getSeconds(), 110, 200);
}

int totalSeconds =0;
int totalMinutes  =0;
void setSeconds(int milliseconds)
{
  totalSeconds = (milliseconds / 1000) % 60 ;//millis/1000;
}
int getSeconds()
{
  return totalSeconds;
}
void setMinutes(int milliseconds)
{   
  totalMinutes = ((milliseconds / (1000*60)) % 60);// (seconds/1000)/60;
}
int getMinutes()
{
  return totalMinutes;
}
\end{lstlisting}