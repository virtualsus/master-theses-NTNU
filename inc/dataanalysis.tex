\chapter{Data presentation and analysis}
\label{chap:dataanalysis}

%kids > curious about leap motion. "what is that thing" > seemed to be interested

%theory
%data
%proving the argument

%quantitative data > age, l/r hand, f/m, time, hits
%qualitative data > reaction, performance, other
The usability experiment was carried out on 8 from 10 planed participants. The number of participants was decreased because the test did not provide any new relevant data. Furthermore, the usability experiment focused more on gathering qualitative data than on quantitative data collection. Additionally, the youngest participant was 3 years old which deviates from the planed minimum age of 2 years. The age and gender distribution of the usability study is presented in table \ref{tab:participanttable}.

%all were right handed and novice users


 \begin{table}[h]
     \centering
     \begin{tabular}{c|c|c|c|c|c|c|c}
     \hline
        \multicolumn{1}{|l|}{\textbf{AGE:}}  &
        \multicolumn{1}{l|}{3}  &     
        \multicolumn{1}{l|}{4}  & 
        \multicolumn{1}{l|}{5}  & 
        \multicolumn{1}{l|}{7}  & 
        \multicolumn{1}{l|}{8}  & 
        \multicolumn{1}{l|}{9}  & 
        \multicolumn{1}{l|}{13} \\ \hline
        \multicolumn{1}{|l|}{\textbf{AMOUNT:}} &
        \multicolumn{1}{l|}{1}  &
        \multicolumn{1}{l|}{1}  &
        \multicolumn{1}{l|}{1}  &
        \multicolumn{1}{l|}{2}  &
        \multicolumn{1}{l|}{1}  &
        \multicolumn{1}{l|}{1}  &
        \multicolumn{1}{l|}{1}  \\ \hline
        \multicolumn{1}{|l|}{\textbf{GENDER:}}  &
        \multicolumn{1}{|l|}{f} &
        \multicolumn{1}{l|}{m}  &
        \multicolumn{1}{l|}{m}  &
        \multicolumn{1}{l|}{f/m}&
        \multicolumn{1}{l|}{f}  &
        \multicolumn{1}{l|}{m}  &
        \multicolumn{1}{l|}{f}  \\ \hline
        \multicolumn{1}{|l|}{\textbf{L/R HANDED:}} &
        \multicolumn{1}{|l|}{R} &
        \multicolumn{1}{l|}{R}  &
        \multicolumn{1}{l|}{R}  &
        \multicolumn{1}{l|}{R}  &
        \multicolumn{1}{l|}{R}  &
        \multicolumn{1}{l|}{R}  &
        \multicolumn{1}{l|}{R}  \\ \hline
     \end{tabular}
     \caption{Presentation of participants}
     \label{tab:participanttable}
 \end{table}
 
 

The age of the participant ranged from three to 13 years and included a mixture of boys and girls. 50\texttt{\%} boys and 50\texttt{\%} girls was a good balanced gender distribution. The statistical calculation in table  \ref{tab:agestatistic} illustrates the average age of this group.

\begin{table}[h]
    \centering
    \begin{tabular}{c|c|c|c}
    \hline
        \multicolumn{1}{|l|}{MEAN} &
        \multicolumn{1}{l|}{MEDIAN} &
        \multicolumn{1}{l|}{MODE} &
        \multicolumn{1}{l|}{STANDARD DEVIATION}\\ \hline
        \multicolumn{1}{|c|}{7} &
        \multicolumn{1}{c|}{7} &
        \multicolumn{1}{c|}{7} &
        \multicolumn{1}{c|}{2,96} \\ \hline
    \end{tabular}
    \caption{Average age of participants}
    \label{tab:agestatistic}
\end{table}

%mean:7 
%median: 7
%mode: 7
%standard deviation: 2,96
%Age from 3 - 13
%Gender 4 boys 4 girls
%13
%8
%4
%7
%7
%5
%9
%3

%Paramenters

%7 of 8 finished the game
%6 of 8 understood the game
%

