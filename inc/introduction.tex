\chapter{Introduction}
\label{chap:introduction}

Conventional user interfaces in our time are mainly focusing on input from different devices such as mouse, keyboard or touchscreen. Many science fiction movies introduce the concept of gestural interaction on touchless interfaces where hand and body motions perform one or other kind of actions to interact with the system. With the proliferation of mobile phones, motion tracking and sensor technology gestural interaction has become a subject of undergoing intense study. Researcher such as Montero et al claim that gestural interfaces are a more natural method of interaction (Montero et al., 2010). However, to what extend gestural interfaces are natural interactions is an issue where opinions are diverged greatly. Malizia and Norman for instance, stated that as long as people need to learn gesture command the interaction is not natural. The advantages are enormous in situations where natural manipulation is required, shortcut interactions are needed, a danger of distraction of visual attention and where interactions must be flexible because of various users (people with disabilities) and environments (Park and Han, 2014). Riener implied that research showed that gestural interactions decrease the cognitive and visual workload (Riener, 2012). Furthermore, as a relatively new modality, gestural interfaces seems to positively stimulate the feeling and behaviour of people and gives them joy and pleasure (Loehmann et al., 2013) (Ren et al., 2013). However, according to Freeman, gestural interfaces require the users full mental and physical attention (Dustin Freeman, 2012) and others researchers are even more sceptical. They believe that gestural interface are a step backward in usability of interfaces due to the lack of guidelines for gestural control. Interaction design principles such as visibility, consistency, discoverability, feedback and recovery are more or less absence  (Norman and Nielsen, 2010). Furthermore, the importance of social acceptance is also an issue that needs to be contemplated when designing gesture-based interfaces (Montero et al., 2010). Since there are no principles for gesture interfaces regarding social acceptance, researcher have proposed some guidelines including naturalness, unobtrusiveness and discreetness. (Montero et al., 2010) 




