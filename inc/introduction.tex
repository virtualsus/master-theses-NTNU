\chapter{Introduction}
\label{chap:introduction}


Conventional user interfaces in our time are mainly focusing on input from different devices such as mouse, keyboard or touchscreen. Many science fiction movies introduce the concept of gestural interaction on touch-less interfaces where hand and body motions perform one or other kind of actions to interact with the system. With the proliferation of mobile phones, motion tracking and sensor technology gestural interaction has become a subject of undergoing intense study. Researcher such as Montero et al. claim that gestural interfaces are a more natural method of interaction \cite{Montero2010}. However, to what extend gestural interfaces are natural interactions is an issue where opinions are diverged greatly \cite{Norman2010}. Malizia and Norman \cite{Malizia2012, Norman2010} for instance, stated that as long as people need to learn gesture command the interaction is not natural. The advantages are enormous in situations where natural manipulation is required, shortcut interactions are needed, a danger of distraction of visual attention and where interactions must be flexible because of various users (people with disabilities) and environments \cite{ParkWonkyu2014}. Riener implied that research showed that gestural interactions decrease the cognitive and visual workload \cite{Riener2012}. Furthermore, as a relatively new modality, gestural interfaces seems to positively stimulate the feeling and behaviour of people and gives them joy and pleasure \cite{Loehmann2013, RenGang2013}. However, according to Freeman, gestural interfaces require the users full mental and physical attention \cite{Freeman2012} and others researchers are even more skeptical. They believe that gestural interface are a step backward in usability of interfaces due to the lack of guidelines for gestural control. Interaction design principles such as visibility, consistency, discoverability, feedback and recovery are more or less absence \cite{NormanNielsen2010}. Furthermore, the importance of social acceptance is also an issue that needs to be contemplated when designing gesture-based interfaces \cite{Montero2010}. Since there are no principles for gesture interfaces regarding social acceptance, researcher have proposed some guidelines including naturalness, unobtrusiveness and discreetness \cite{Montero2010}. 
\newline
In order to develop useful gesture interface or gesture based systems, meaningful gestures are essential. However, not less important are suitable sensors that satisfy the requirements of the gesture-based system.
Gestural sensor technology is a growing industry and many various devices have been already developed. Oculus Rift\footnote{https://www.oculus.com/} for instance, is very popular and well known in the VR industry. Unfortunately, this device is quite pricey and it requires handhold devices to interact. However, there are other devices that able to record hand and body postures and gestures without any handhold aid.
These can be divided in wearable such as data-gloves \cite{Gupta2001} and hand-free such as Kinect and Leap Motion.
The question is whether those devices can be used in a medical context, specifically as a screening method for motor development. Nowadays, method for evaluating development delays are neither digital or computer based. A digital solution would decrease data processing time and be more reliable since the gained data is less subjective. Autism Spectrum Disorder is used as an example to illustrate potential application areas. Researchers such as Anzulewicz investigate the motor performance on children with ASD and she pointed out that there is a need computational system to measure motor-performance \cite{Anzulewicz2016}. 
\newline
To summarize, the aim of this study is to investigate whether hand-free interaction can be used as a new screening method for development delay in children.
\newline
The first step of the project is to gather general information. The literature review (chap. \ref{chap:background}) will enlighten the background theory in order to gain a deeper understanding of the topic. Additionally, the review will provide an insight of previous work done in this field. This will be the framework and guideline for the other chapter where the aim of the study is elaborated and research questions and hypotheses are formulated. The method chapter (chap. \ref{chap:methods}) describes the process of designing the prove of principles, the approach and materials that were used during the design process. The gathered data is then presented and analyzed in chapter \ref{chap:dataanalysis}. Finally, the findings are compared with the theory from chapter \ref{chap:background} and discussed further in the discussion chapter (\ref{chap:discussion}). The discussion will enlighten whether the hypotheses can be confirmed or whether it must be refuted.


%what is the topic
%Explain why this topic is important
%brief summary of previous work
%outlines the purpose
%raod map> contents of each chap
%!Never put any results or decisions in the Introduction!

%things to mention in intro
%motor performance delay e.g. in asd
%touch-free sensors
%development screening methods
%performance measure
%gesture interfaces
%gesture interction


%To develop a gesture interface, we need some criteria to evaluate its performance,such as: meaningful gestures; suitable sensors; efficient training algorithms; and accurate,efficient, on-line/real-time recognition. 
%the aim of the study is to..

\section{Justification, motivation and benefits}
Gestural touch-less interfaces as a relatively new form of interaction is an exciting field of study. Using this new form of interaction in health care and medical science can open doors for new ideas and possibilities. For that reason it is worth to investigate the feasibility of creating a gesture based system to extract parameters which can be processed and analyzed in order to detect deviation from normal motor development in children. Data gained during the analyzing process could than be used as an indicator for development delays such as in Autism Spectrum Disorder.
Moreover, if the study reveals that a portable device such as Leap Motion is suitable for extracting important parameter regarding development delays then it might be an interesting and affordable alternative to nowadays screening methods.



